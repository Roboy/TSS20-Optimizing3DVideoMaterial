\chapter{Outlook}\label{chapter:outlook}
There are several possibilities for further improvements. One of them is that the predicted gaze needs some time to the server where the foveated rendering is calculated and then returns to the client. It should be determined whether it is necessary to forecast the gaze for some frames to reach the client faster with the current gaze. This requires that forecasting of the gaze is possible.
\par
Several opportunities exist for improving the latency. The first is to check whether a traditional video codec for compression is required. The encoding and decoding consume most of the latency and is thus the best option for improvements. Facebook uses a spatial matrix for its DeepFovea, which cannot be transmitted by a traditional video platform. Realising a custom transmission protocol to send parts of the matrix would result in more lossy quality but a much faster speed. The matrix can be composed back to an image with the help of a neural network performing superresolution. The proposed neural network must thus change its input type, which would also result in fewer conversations. Additionally, the transmission of the gaze coordinates would be improved as the frame can hold the coordinates by itself, instead of the parallel transmission. Another possibility is to use a neural network codec, which has been proposed recently. These types of codecs also promise to improve the latency by keeping the same quality level, as determined by \cite{Ma2020}.
\par
Moreover, better hardware for floating-point operations, such as the newest GPU generation of Nvidia, promises better machine learning support and faster transcoding \parencite{RTX3000}.