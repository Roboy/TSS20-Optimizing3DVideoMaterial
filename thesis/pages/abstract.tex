\chapter{\abstractname}
Real-time streaming for telepresence is with current streaming challenging. Real-time streaming becomes essential as with the recent developments in virtual reality, enchanting applications for telepresence are becoming possible. For controlling robots or other time-sensitive tasks, virtual reality represents the robot's view of the natural environment to the user in 3D. However, the robot may be all around the world. One of the main problems is that the user suffers side-effects and cannot control the robot appropriately when the camera stream delays from the robot to the head-mounted display. In this context, telepresence is defined as the status of feeling to be in a distant environment. The higher the immersion of the virtual reality is, the more the user feels to be in that other location. 
\par
This project aims to develop a low latency streaming application that adjusts the stream dynamically to the available bandwidth to provide the best possible image to the user. The difference as compared to streaming used by Netflix or television is that the low latency requirement while having only low bandwidth available in real-time is in a tremendous contrast as low latency requires less computation resulting in a higher transfer rate. However, with the upcoming availability, for example by the HTC VIVE Pro Eye, which predicts the current user's gaze in virtual reality, the transfer rate is reduced to a minimum by a technique named foveated rendering. The solution includes multiple steps: downscaling the video by calculating a foveated rendering, transmitting the video, performing superresolution of the frames of the streamed video, and delivering it to a head-mounted display showing the virtual reality. In this presented application, calculating the foveated rendering is achieved by splitting the foveated area and the peripheral region into two streams with high-resolution and low-resolution. Machine learning performs superresolution of the received background at the client-side to increase the background details. Afterwards, the foveated and the peripheral area are merged and shown to the user. 
\par
The results showed an implementation, which reduces the latency with foveated rendering being considerably lower in comparison to traditional streaming solutions. Furthermore, the quality in the current user's gaze is high when the available bandwidth is low. These results suggest that for performing time-sensitive tasks requiring telepresence, the concept of streaming including foveated rendering and superresolution should be taken into account.